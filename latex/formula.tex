\documentclass[UTF8,12pt]{ctexart}
\usepackage{amsmath}
\usepackage{color}
\definecolor{myback}{RGB}{67,67,67}
\pagecolor{myback}

\begin{document}
\textcolor{white}{\LARGE Einstein 's $E=mc^2$.}

\[ E=mc^2. \]

\begin{equation}
E=mc^2.
\end{equation}

\[ \sum\limits_{i = 1}^{n} x_i = x_1 + x_2 + \dots + x_n\]

\[ \prod\limits_{i = 1}^{n} x_i = x_1 \times x_2 \times \dots \times x_n\]

\[ \int_{0}^{T}f(t)dt\]

\[ \alpha,\beta \]

\[ \gamma \]
\[ \Gamma \]
\[ X \sim \Gamma (\alpha,\beta) \]

\[ \delta \]
\[ \Delta \]

\[ \Delta f(x_k) = f(x_{k+1}) - f(x_k) \]

%设 $\{x_n\},x_n \in R,n = 1,2,..., x_0 \in R$,对于任意的正实数$\epsilon$,存在自然数N,使得当$n > N$时,有$\left| x_n - x_0 \right| < \epsilon$

设 $\{x_n\}$为一数列,如果存在常数a,对于任意给定的正数$\epsilon$(不论它多么小),总存在正整数N,使得当$n > N$时,不等式$\left| x_n - a \right| < \epsilon$都成立


设 $x=\eta_1$、$x=\eta_2$是方程组 $Ax=b$ 的解,则$x=\eta_1 - \eta_2$是对应的齐次线性方程组$Ax=0$的解。

\[ f(x_1,x_2 | \theta) = f(x_1 | \theta) \times f(x_2 | \theta)\]

\[ Av = \lambda v \]

设$x=\eta$是方程组$Ax=b$的解,$x=\xi$是对应的齐次线性方程组$Ax=0$的解,则$x=\eta + \xi$仍是方程组$Ax=b$的解。

一般,若参数方程:

\[
\begin{cases}
x =  \phi (t) \\
y =  \psi (t) \\
\end{cases}
\]

确定y与x间的函数关系,则称此函数关系所表达的函数为由上面的参数方程所确定的函数

%设$\Omega$是随机试验$E$的样本空间,x是任意实数,称函数$F(x) = P\{ X \leqslant x\} = P\{\omega: X(\omega) \leqslant x \}$
%为随机变量$X$的分布函数

设$\Omega$是随机试验$E$的样本空间,x是任意实数,称函数:
\[F(x)= P\{ X \le x\} = P\{\omega: X(\omega) \le x \}\]为随机变量$X$的分布函数


\[ X \sim \chi^2(k) \]


\end{document}